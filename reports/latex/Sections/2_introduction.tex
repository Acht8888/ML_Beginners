\section{Introduction}
\subsection{Background}
Customer churn prediction is a crucial task in understanding why customers discontinue their subscriptions and how businesses can mitigate these losses. In the telecommunications industry, customer churn directly impacts revenue and customer retention strategies. By analyzing customer behavior, service usage patterns, and contractual details, businesses can take proactive measures to enhance customer experience and reduce churn rates. This project focuses on building and comparing multiple machine learning models, including decision trees, neural networks, naive Bayes, genetic algorithms, and graphical models, to predict customer churn in the telecom sector. The performance of these models is evaluated using precision, recall, and F1-score to determine their effectiveness in identifying potential churners.

\subsection{Goal}
The primary goal of this project is to implement and compare various machine learning models to understand their strengths and weaknesses in predicting customer churn. Rather than focusing on achieving state-of-the-art performance, this study emphasizes model engineering, implementation, and comparative analysis. By evaluating different algorithms, we aim to gain hands-on experience and insights into their practical applications in churn prediction.

\subsection{Setup}
The project is implemented using PyTorch for model development and training. The Telco Customer Churn dataset is preprocessed to handle missing values, encode categorical features, and normalize numerical data. Various machine learning models are then trained and evaluated using standard classification metrics. The experimental setup includes computational resource considerations, hyperparameter tuning, and performance analysis across different models.
