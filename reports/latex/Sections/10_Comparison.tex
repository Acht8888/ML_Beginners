\section{Comparison}
In this section we will compare five machine learning models — Decision Tree, Neural Network, Naive Bayes, Genetic Algorithm, and Graphical Model — applied to the Telco dataset for churn prediction. The comparison is based on key performance metrics, computational efficiency, and overall model interpretability.\\

The Decision Tree model is known for its interpretability and simplicity. It excels at capturing non-linear relationships by recursively splitting the data into subsets based on the most significant features. In the Telco dataset, the Decision Tree likely performed well due to its ability to handle categorical data efficiently and pinpoint key decision points leading to churn. However, Decision Trees are prone to overfitting, especially if the tree grows too deep. Pruning techniques and limiting tree depth help mitigate this, balancing performance and generalization. While the Decision Tree provides insights into the most influential factors behind customer churn, it may struggle when patterns become highly complex or data imbalance arises.\\

The Neural Network model offers a robust approach for capturing intricate patterns in the data. With multiple hidden layers and activation functions, it learns complex representations that other models might overlook. In this context, the Neural Network demonstrated strong predictive performance, particularly excelling in scenarios where relationships between features were non-linear and multi-faceted. However, this increased complexity comes at the cost of interpretability and computational efficiency. Neural Networks require careful hyperparameter tuning, and they can overfit without sufficient regularization. Despite these challenges, the model’s ability to generalize across diverse scenarios makes it well-suited for churn prediction, especially in datasets with nuanced behaviors like Telco.\\

The Naive Bayes model, despite its simplistic assumption of feature independence, delivered competitive results. Its strength lies in its efficiency and ability to handle categorical features naturally. In this study, tuning the Laplace smoothing parameter improved performance by handling unseen values in the test set effectively. The Categorical Naive Bayes variant leveraged categorical features well, while the Random Forest implementation of Naive Bayes added ensemble learning, improving robustness. However, Naive Bayes can falter when feature independence assumptions don’t hold — a likely scenario in churn prediction where multiple factors interplay. Nonetheless, its speed and simplicity make it a valuable contender.\\

The Genetic Algorithm presented a novel approach by simulating evolution through selection, crossover, and mutation to optimize predictions. This model's adaptability allowed it to explore diverse feature combinations, identifying unique patterns that other models might miss. However, Genetic Algorithms are computationally intensive, requiring multiple generations to converge toward optimal solutions. In the Telco dataset, its performance was promising, especially when fine-tuned, but the trade-off between computational complexity and predictive gains must be carefully managed. The method stands out when traditional models struggle, offering a fresh perspective on feature selection and optimization.\\

Lastly, the Graphical Model leveraged probabilistic relationships between features, capturing dependencies often overlooked by simpler models. By modeling the conditional dependencies between variables, it provided a structured way to reason about customer churn. Graphical Models shine when understanding relationships between features is as crucial as prediction itself. In this case, it delivered a well-rounded performance, balancing interpretability and predictive accuracy. However, constructing the graph structure and estimating probabilities require domain knowledge and careful design, making it less accessible than other methods.\\

In conclusion, each model brought unique strengths to the table. The Neural Network stood out for its capacity to handle complex patterns, while the Decision Tree provided interpretability. Naive Bayes proved efficient with categorical features, and the Genetic Algorithm offered innovative optimization strategies. The Graphical Model provided a structured view of feature dependencies, enhancing interpretability. The optimal model choice depends on the balance between performance, interpretability, and computational efficiency for deployment in a real-world churn prediction system.